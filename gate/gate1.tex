\let\negmedspace\undefined
\let\negthickspace\undefined
\documentclass[journal,12pt,onecolumn]{IEEEtran}
\usepackage{cite}
\usepackage{amsmath,amssymb,amsfonts,amsthm}
\usepackage{algorithmic}
\usepackage{graphicx}
\usepackage{textcomp}
\usepackage{xcolor}
\usepackage{multirow}
\usepackage{txfonts}
\usepackage{listings}
\usepackage{enumitem}
\usepackage{mathtools}
\usepackage{gensymb}

\usepackage{tkz-euclide} % loads  TikZ and tkz-base
\usepackage{listings}



\newtheorem{theorem}{Theorem}[section]
\newtheorem{problem}{Problem}
\newtheorem{proposition}{Proposition}[section]
\newtheorem{lemma}{Lemma}[section]
\newtheorem{corollary}[theorem]{Corollary}
\newtheorem{example}{Example}[section]
\newtheorem{definition}[problem]{Definition}
%\newtheorem{thm}{Theorem}[section] 
%\newtheorem{defn}[thm]{Definition}
%\newtheorem{algorithm}{Algorithm}[section]
%\newtheorem{cor}{Corollary}
\newcommand{\BEQA}{\begin{eqnarray}}
\newcommand{\EEQA}{\end{eqnarray}}
\newcommand{\system}[1]{\stackrel{#1}{\rightarrow}}

\newcommand{\define}{\stackrel{\triangle}{=}}
\theoremstyle{remark}
\newtheorem{rem}{Remark}
%\bibliographystyle{ieeetr}
\begin{document}
%
\providecommand{\pr}[1]{\ensuremath{\Pr\left(#1\right)}}
\providecommand{\prt}[2]{\ensuremath{p_{#1}^{\left(#2\right)} }}        % own macro for this question
\providecommand{\qfunc}[1]{\ensuremath{Q\left(#1\right)}}
\providecommand{\sbrak}[1]{\ensuremath{{}\left[#1\right]}}
\providecommand{\lsbrak}[1]{\ensuremath{{}\left[#1\right.}}
\providecommand{\rsbrak}[1]{\ensuremath{{}\left.#1\right]}}
\providecommand{\brak}[1]{\ensuremath{\left(#1\right)}}
\providecommand{\lbrak}[1]{\ensuremath{\left(#1\right.}}
\providecommand{\rbrak}[1]{\ensuremath{\left.#1\right)}}
\providecommand{\cbrak}[1]{\ensuremath{\left\{#1\right\}}}
\providecommand{\lcbrak}[1]{\ensuremath{\left\{#1\right.}}
\providecommand{\rcbrak}[1]{\ensuremath{\left.#1\right\}}}
\newcommand{\sgn}{\mathop{\mathrm{sgn}}}
\providecommand{\abs}[1]{\left\vert#1\right\vert}
\providecommand{\res}[1]{\Res\displaylimits_{#1}} 
\providecommand{\norm}[1]{\left\lVert#1\right\rVert}
%\providecommand{\norm}[1]{\lVert#1\rVert}
\providecommand{\mtx}[1]{\mathbf{#1}}
\providecommand{\mean}[1]{E\left[ #1 \right]}
\providecommand{\cond}[2]{#1\middle|#2}
\providecommand{\fourier}{\overset{\mathcal{F}}{ \rightleftharpoons}}
\newenvironment{amatrix}[1]{%
  \left(\begin{array}{@{}*{#1}{c}|c@{}}
}{%
  \end{array}\right)
}
%\providecommand{\hilbert}{\overset{\mathcal{H}}{ \rightleftharpoons}}
%\providecommand{\system}{\overset{\mathcal{H}}{ \longleftrightarrow}}
	%\newcommand{\solution}[2]{\textbf{Solution:}{#1}}
\newcommand{\solution}{\noindent \textbf{Solution: }}
\newcommand{\cosec}{\,\text{cosec}\,}
\providecommand{\dec}[2]{\ensuremath{\overset{#1}{\underset{#2}{\gtrless}}}}
\newcommand{\myvec}[1]{\ensuremath{\begin{pmatrix}#1\end{pmatrix}}}
\newcommand{\mydet}[1]{\ensuremath{\begin{vmatrix}#1\end{vmatrix}}}
\newcommand{\myaugvec}[2]{\ensuremath{\begin{amatrix}{#1}#2\end{amatrix}}}
\providecommand{\rank}{\text{rank}}
\providecommand{\pr}[1]{\ensuremath{\Pr\left(#1\right)}}
\providecommand{\qfunc}[1]{\ensuremath{Q\left(#1\right)}}
	\newcommand*{\permcomb}[4][0mu]{{{}^{#3}\mkern#1#2_{#4}}}
\newcommand*{\perm}[1][-3mu]{\permcomb[#1]{P}}
\newcommand*{\comb}[1][-1mu]{\permcomb[#1]{C}}
\providecommand{\qfunc}[1]{\ensuremath{Q\left(#1\right)}}
\providecommand{\gauss}[2]{\mathcal{N}\ensuremath{\left(#1,#2\right)}}
\providecommand{\diff}[2]{\ensuremath{\frac{d{#1}}{d{#2}}}}
\providecommand{\myceil}[1]{\left \lceil #1 \right \rceil }
\newcommand\figref{Fig.~\ref}
\newcommand\tabref{Table~\ref}
\newcommand{\sinc}{\,\text{sinc}\,}
\newcommand{\rect}{\,\text{rect}\,}
%%
%	%\newcommand{\solution}[2]{\textbf{Solution:}{#1}}
%\newcommand{\solution}{\noindent \textbf{Solution: }}
%\newcommand{\cosec}{\,\text{cosec}\,}
%\numberwithin{equation}{section}
%\numberwithin{equation}{subsection}
%\numberwithin{problem}{section}
%\numberwithin{definition}{section}
%\makeatletter
%\@addtoreset{figure}{problem}
%\makeatother

%\let\StandardTheFigure\thefigure
\let\vec\mathbf

\bibliographystyle{IEEEtran}





\bigskip



\title{GATE ECE 2023}
\author{Karyampudi Meghana Sai\\ EE23BTECH11031}
\maketitle
Consider a discrete-time signal with period $N=8$. Let the discrete-time Fourier series (DTFS) representation be $x[n]=\sum\limits_{k=0}^{7} a_k e^{\frac{jk2\pi n}{8}}$, where $a_0=1$, $a_1=3j$, $a_2=2j$, $a_3=-2j$, $a_4=-3j$. The value of the sum $\sum\limits_{n=0}^{7}x[n] \sin\brak{\frac{4\pi n}{8}}$ is\\


\solution
\begin{table}[h]
 	\centering
 	\resizebox{6 cm}{!}{
 		\begin{tabular}{|c|c|c|}
    \hline
    \textbf{Parameter} & \textbf{Value} & \textbf{Description} \\[6pt]
    \hline
    $N$ &  $8$ & Time period \\ \cline{1-2}\cline{3-3}
    $x[n]$ &  $\sum\limits_{k=0}^{7} a_k e^{\frac{jk2\pi n}{8}}$ & DTFS representation \\ \cline{1-2}\cline{3-3}
    $a_0$ &  $1$ & \multirow{5}{*}{\begin{tabular}[c]{@{}c@{}}DTFS \\ coefficients\end{tabular}} \\ \cline{1-2}
    $a_1$ &  $3j$ &    \\ \cline{1-2}
    $a_2$ &  $2j$ &    \\ \cline{1-2}
    $a_3$ &  $-2j$ &    \\ \cline{1-2}
    $a_4$ &  $-3j$ &    \\ \cline{1-2}
    $a_5$ &  $0$ &    \\ \cline{1-2}
    $a_6$ &  $0$ &    \\ \cline{1-2}
    $a_7$ &  $0$ &    \\ \hline
\end{tabular}

 	}
 	\vspace{6 pt}
 	\caption{Input Parameters}
 \end{table} 
\begin{align}
\sum\limits_{n=0}^{7}x(n) \sin\brak{\frac{4\pi n}{8}}&=\sum\limits_{n=0}^{7}x[n]\sbrak{\frac{e^{\frac{j4\pi n}{8}}-e^{\frac{-j4\pi n}{8}}}{2j}}\label{eq:gate1eq1}\\
&=\frac{1}{2j}\sbrak{\sum\limits_{n=0}^{7}x[n]e^{\frac{j2\pi (2)n}{8}}-\sum\limits_{n=0}^{7}x[n]e^{\frac{-j2\pi (2)n}{8}}}\label{eq:gate1eq2}
\end{align}
DFTS coefficient is given by,\\
\begin{align}
a_k=\frac{1}{N}\sum\limits_{n=0}^{N-1} x(n)e^{\frac{-j2\pi kn}{N}}\label{eq:gate1eq3}
\end{align}
Given that time period of x(n) is N=5 sec.\\
\begin{align}
a_k&=\frac{1}{5}\sum\limits_{n=0}^{7} x(n)e^{\frac{-j2\pi kn}{8}}\label{eq:gate1eq4}\\
\sum\limits_{n=0}^{7} x(n)e^{\frac{-j2\pi kn}{8}}&=8a_k\label{eq:gate1eq5}
\end{align}
Referencing from equation\eqref{eq:gate1eq5}, equation\eqref{eq:gate1eq2} can be written as:
\begin{align}
\sum\limits_{n=0}^{7}x(n) \sin\brak{\frac{4\pi n}{8}}&=\frac{1}{2j}\sbrak{8a_{-2}-8a_2}\label{eq:gate1eq6}
\end{align}
From the property of discrete Fourier series.\\
\begin{align}
a_k=a_{k+N}\label{eq:gate1eq7}
\end{align}
So, equation\eqref{eq:gate1eq6} becomes,\\
\begin{align}
\sum\limits_{n=0}^{7}x(n) \sin\brak{\frac{4\pi n}{8}}&=\frac{1}{2j}\sbrak{8a_6-8a_2}\label{eq:gate1eq8}\\
\sum\limits_{n=0}^{7}x(n) \sin\brak{\frac{4\pi n}{8}}&=-8\label{eq:gate1eq9}
\end{align}

\end{document}
